% Ludwik Ciechański
\documentclass[12pt]{article}
    
\usepackage[T1]{fontenc}
\usepackage{beramono}
\usepackage{polski}
\usepackage[utf8]{inputenc}
\usepackage[a4paper, total={6in, 10in}]{geometry}
\usepackage{listings}
\usepackage[usenames,dvipsnames]{xcolor}
\usepackage{graphicx}
\usepackage{float}
\usepackage{color}
\usepackage{titlesec}
\usepackage[hidelinks]{hyperref}
\usepackage[font=small]{caption}

\newcommand{\cFaculty}[1]{\Large \textsc{#1}\par}
\newcommand{\cSubject}[1]{\Large #1\par}
\newcommand{\cTitle}[1]{\LARGE \textbf{#1}\par}
\newcommand{\cAuthors}[1]{\large #1\par}
\newcommand{\cDate}[1]{\small #1\par}

\setcounter{section}{-1}
\graphicspath{ {images/} }
\titlespacing{\section}{0pt}{2pt}{2pt}
\titlespacing{\subsection}{0pt}{3pt}{2pt}

%%
%% Julia definition (c) 2014 Jubobs
%%
\lstdefinelanguage{Julia}
  {morekeywords={abstract,break,case,catch,const,continue,do,else,elseif,
      end,export,false,for,function,immutable,import,importall,if,in,
      macro,module,otherwise,quote,return,switch,true,try,type,typealias,
      using,while},
   sensitive=true,
   alsoother={$},
   morecomment=[l]\#,
   morecomment=[n]{\#=}{=\#},
   morestring=[s]{"}{"},
   morestring=[m]{'}{'},
}[keywords,comments,strings]

\lstset{
  language         = Julia,
  basicstyle       = \ttfamily,
  keywordstyle     = \bfseries\color{blue},
  stringstyle      = \color{magenta},
  commentstyle     = \color{ForestGreen},
  showstringspaces = false,
  aboveskip=3mm,
  belowskip=3mm,
  numbers=none,
  breaklines=true,
  breakatwhitespace=true,
  tabsize=4,
  frame=shadowbox
}
